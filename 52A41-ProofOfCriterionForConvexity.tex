\documentclass[12pt]{article}
\usepackage{pmmeta}
\pmcanonicalname{ProofOfCriterionForConvexity}
\pmcreated{2013-03-22 17:00:23}
\pmmodified{2013-03-22 17:00:23}
\pmowner{rspuzio}{6075}
\pmmodifier{rspuzio}{6075}
\pmtitle{proof of criterion for convexity}
\pmrecord{5}{39288}
\pmprivacy{1}
\pmauthor{rspuzio}{6075}
\pmtype{Proof}
\pmcomment{trigger rebuild}
\pmclassification{msc}{52A41}
\pmclassification{msc}{26A51}
\pmclassification{msc}{26B25}

% this is the default PlanetMath preamble.  as your knowledge
% of TeX increases, you will probably want to edit this, but
% it should be fine as is for beginners.

% almost certainly you want these
\usepackage{amssymb}
\usepackage{amsmath}
\usepackage{amsfonts}

% used for TeXing text within eps files
%\usepackage{psfrag}
% need this for including graphics (\includegraphics)
%\usepackage{graphicx}
% for neatly defining theorems and propositions
\usepackage{amsthm}
% making logically defined graphics
%%%\usepackage{xypic}

% there are many more packages, add them here as you need them

% define commands here
\newtheorem{thm}{Theorem}
\begin{document}
\begin{thm}
Suppose $f \colon (a,b) \to \mathbb{R}$ is continuous and that,
for all $x,y \in (a,b)$,
\[
f \left( {x + y \over 2} \right) \le {f(x) + f(y) \over 2} .
\]
Then $f$ is convex.
\end{thm}

\begin{proof}
We begin by showing that, for any natural numbers $n$ and $m \le 2^n$,
\[
f \left( {m x + (2^n - m) y \over 2^n} \right) \le 
{m f(x) + (2^n - m) f(y) \over 2^n} 
\]
by induction.  When $n=1$, there are three possibilities: $m=1$, $m=0$,
and $m=2$.  The first possibility is a hypothesis of the theorem being
proven and the other two possibilities are trivial.

Assume that
\[
f \left( {m' x + (2^n - m') y \over 2^n} \right) \le 
{m' f(x) + (2^n - m') f(y) \over 2^n} 
\]
for some $n$ and all $m' \le 2^n$.  Let $m$ be a number less than or equal
to $2^{n+1}$.  Then either $m \le 2^n$ or $2^{n+1} - m \le 2^n$.  In the
former case we have
\begin{align*}
f \left( {1 \over 2} {m x + (2^n - m) y \over 2^n} + 
{y \over 2} \right) &\le 
{1 \over 2} \left( f \left( {m x + (2^n - m) y \over 2^n} \right) +
f(y) \right) \\ &\le
{1 \over 2} {m f(x) + (2^n - m) f(y) \over 2^n} +
{f(y) \over 2} \\ &=
{m f(x) + (2^{n+1} - m) f(y) \over 2^{n+1}} .
\end{align*}
In the other case, we can reverse the roles of $x$ and $y$.

Now, every real number $s$ has a binary expansion; in other words, there
exists a sequence $\{m_n\}$ of integers such that $\lim_{n \to \infty} 
m_n / 2^n = s$.  If $0 \le s \le 1$, then we also have $0 \le m_n \le 2^n$
so, by what we proved above,
\[
f \left( {m_n x + (2^n - m_n) y \over 2^n} \right) \le 
{m_n f(x) + (2^n - m_n) f(y) \over 2^n} .
\]
Since $f$ is assumed to be continuous, we may take the limit of both
sides and conclude
\[
f( s x + (1 - s) y ) \le s f (x) + (1 - s) f (y) ,
\]
which implies that $f$ is convex.
\end{proof}
%%%%%
%%%%%
\end{document}
