\documentclass[12pt]{article}
\usepackage{pmmeta}
\pmcanonicalname{WulffTheorem}
\pmcreated{2013-03-22 15:19:50}
\pmmodified{2013-03-22 15:19:50}
\pmowner{paolini}{1187}
\pmmodifier{paolini}{1187}
\pmtitle{Wulff theorem}
\pmrecord{8}{37144}
\pmprivacy{1}
\pmauthor{paolini}{1187}
\pmtype{Theorem}
\pmcomment{trigger rebuild}
\pmclassification{msc}{52A21}
%\pmkeywords{isoperimetric}
%\pmkeywords{Frank}
%\pmkeywords{Wulff}
%\pmkeywords{Finsler}
%\pmkeywords{anisotropic}
\pmrelated{FinslerGeometry}
\pmdefines{Wulff shape}

% this is the default PlanetMath preamble.  as your knowledge
% of TeX increases, you will probably want to edit this, but
% it should be fine as is for beginners.

% almost certainly you want these
\usepackage{amssymb}
\usepackage{amsmath}
\usepackage{amsfonts}

% used for TeXing text within eps files
%\usepackage{psfrag}
% need this for including graphics (\includegraphics)
%\usepackage{graphicx}
% for neatly defining theorems and propositions
\usepackage{amsthm}
% making logically defined graphics
%%%\usepackage{xypic}

% there are many more packages, add them here as you need them

% define commands here
\newcommand{\R}{\mathbb R}
\newtheorem{theorem}{Theorem}
\newtheorem{definition}{Definition}
\theoremstyle{remark}
\newtheorem{example}{Example}
\begin{document}
\begin{definition}[Wulff shape]
Let $\phi\colon\R^n \to [0,+\infty)$ be a non-negative, convex, coercive, positively $1$-homogeneous function. We define the \emph{Wulff shape} relative to $\phi$ as the set
\[
  W_\phi := \{ x \in \R^n \colon \text{$\langle x,y\rangle \le 1$ for all $y$ such that $\phi(y)\le 1$}\} 
\]
(where $\langle \cdot,\cdot\rangle$ is the Euclidean inner product in $\R^n$.)
\end{definition}

\begin{theorem}[Wulff]
Let $\phi\colon\R^n \to [0,+\infty)$ be a non-negative, convex, coercive, $1$-homogeneous function. Given a regular open set $D\subset\R^n$ we consider the following anisotropic surface energy:
\[
  F_\phi(D) = \int_{\partial D} \phi(\nu_D(x))\, d\sigma(x)
\]
where $\nu_D(x)$ is the outer unit normal to $\partial D$, and $\sigma$ is the surface area on $\partial D$.
Then, given any set $D$ with the same volume as $W_\phi$, i.e.\ $|D|=|W_\phi|$, one has $F_\phi(D)\ge F_\phi(W_\phi)$.
Moreover if $|D|=|W_\phi|$ and $F_\phi(D)=F_\phi(W_\phi)$ then $D$ is a translation of $W_\phi$ i.e.\ there exists $v\in R^n$ such that $D=v+W_\phi$.
\end{theorem}
%%%%%
%%%%%
\end{document}
