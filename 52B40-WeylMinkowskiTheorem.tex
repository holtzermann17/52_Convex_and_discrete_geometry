\documentclass[12pt]{article}
\usepackage{pmmeta}
\pmcanonicalname{WeylMinkowskiTheorem}
\pmcreated{2013-03-22 16:59:07}
\pmmodified{2013-03-22 16:59:07}
\pmowner{mps}{409}
\pmmodifier{mps}{409}
\pmtitle{Weyl-Minkowski theorem}
\pmrecord{6}{39262}
\pmprivacy{1}
\pmauthor{mps}{409}
\pmtype{Theorem}
\pmcomment{trigger rebuild}
\pmclassification{msc}{52B40}
\pmsynonym{Weyl--Minkowski theorem}{WeylMinkowskiTheorem}

% this is the default PlanetMath preamble.  as your knowledge
% of TeX increases, you will probably want to edit this, but
% it should be fine as is for beginners.

% almost certainly you want these
\usepackage{amssymb}
\usepackage{amsmath}
\usepackage{amsfonts}

% used for TeXing text within eps files
%\usepackage{psfrag}
% need this for including graphics (\includegraphics)
%\usepackage{graphicx}
% for neatly defining theorems and propositions
\usepackage{amsthm}
% making logically defined graphics
%%%\usepackage{xypic}

% there are many more packages, add them here as you need them

% define commands here
\newtheorem*{theorem*}{Theorem}
\begin{document}
\begin{theorem*}[Weyl--Minkowski]
A subset of Euclidean space is a convex polytope if and only if it is a bounded polyhedron.
\end{theorem*}

\begin{thebibliography}{3}
\bibitem{cite:M}
Minkowski, H. \emph{Allgemeine Lehrs\"atze \"uber die konvexe Polyeder}, Nachr.~Ges.~Wiss., G\"ottingen, 1897, 198--219.
\bibitem{cite:W}
Weyl, H. \emph{Elementare Theorie der konvexen Polyeder}, Comment.~Math.~Helvetici, 1935, 7
\bibitem{cite:Z}
Ziegler, G., \emph{Lectures on polytopes}, Springer-Verlag, 1997.
\end{thebibliography}
%%%%%
%%%%%
\end{document}
