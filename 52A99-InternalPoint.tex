\documentclass[12pt]{article}
\usepackage{pmmeta}
\pmcanonicalname{InternalPoint}
\pmcreated{2013-03-22 14:25:04}
\pmmodified{2013-03-22 14:25:04}
\pmowner{jirka}{4157}
\pmmodifier{jirka}{4157}
\pmtitle{internal point}
\pmrecord{5}{35923}
\pmprivacy{1}
\pmauthor{jirka}{4157}
\pmtype{Definition}
\pmcomment{trigger rebuild}
\pmclassification{msc}{52A99}

\endmetadata

% this is the default PlanetMath preamble.  as your knowledge
% of TeX increases, you will probably want to edit this, but
% it should be fine as is for beginners.

% almost certainly you want these
\usepackage{amssymb}
\usepackage{amsmath}
\usepackage{amsfonts}

% used for TeXing text within eps files
%\usepackage{psfrag}
% need this for including graphics (\includegraphics)
%\usepackage{graphicx}
% for neatly defining theorems and propositions
\usepackage{amsthm}
% making logically defined graphics
%%%\usepackage{xypic}

% there are many more packages, add them here as you need them

% define commands here
\theoremstyle{theorem}
\newtheorem*{thm}{Theorem}
\newtheorem*{lemma}{Lemma}
\newtheorem*{conj}{Conjecture}
\newtheorem*{cor}{Corollary}
\newtheorem*{example}{Example}
\theoremstyle{definition}
\newtheorem*{defn}{Definition}
\begin{document}
\begin{defn}
Let $X$ be a vector space and $S \subset X$.  Then $x \in S$ is called an
{\em internal point} of $S$ if and only if the intersection of each line in $X$ through $x$ and $S$ contains a small interval around $x$.
\end{defn}

That is $x$ is an internal point of $S$ if whenever $y \in X$ there exists an $\epsilon > 0$ such that $x + ty \in S$ for all $t < \epsilon$.

If $X$ is a topological vector space and $x$ is in the interior of $S$, then it is an internal point, but the converse is not true in general.  However if $S \subset {\mathbb{R}}^n$ is a convex set then all internal points are interior points and vice versa.

\begin{thebibliography}{9}
\bibitem{royden}
H.\@ L.\@ Royden. \emph{\PMlinkescapetext{Real Analysis}}. Prentice-Hall, Englewood Cliffs, New Jersey, 1988
\end{thebibliography}
%%%%%
%%%%%
\end{document}
