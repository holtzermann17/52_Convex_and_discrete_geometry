\documentclass[12pt]{article}
\usepackage{pmmeta}
\pmcanonicalname{ProperCone}
\pmcreated{2013-03-22 14:37:13}
\pmmodified{2013-03-22 14:37:13}
\pmowner{dooder0001}{4288}
\pmmodifier{dooder0001}{4288}
\pmtitle{proper cone}
\pmrecord{7}{36199}
\pmprivacy{1}
\pmauthor{dooder0001}{4288}
\pmtype{Definition}
\pmcomment{trigger rebuild}
\pmclassification{msc}{52A20}
\pmrelated{Cone3}
\pmrelated{Cone5}

% this is the default PlanetMath preamble.  as your knowledge
% of TeX increases, you will probably want to edit this, but
% it should be fine as is for beginners.

% almost certainly you want these
\usepackage{amssymb}
\usepackage{amsmath}
\usepackage{amsfonts}

% used for TeXing text within eps files
%\usepackage{psfrag}
% need this for including graphics (\includegraphics)
%\usepackage{graphicx}
% for neatly defining theorems and propositions
%\usepackage{amsthm}
% making logically defined graphics
%%%\usepackage{xypic}

% there are many more packages, add them here as you need them

% define commands here
\begin{document}
A \emph{proper cone} is a \PMlinkname{cone}{Cone3} $C\subset\mathbb{R}^n$ that satisfies the following:
\begin{itemize}
\item $C$ is convex;
\item $C$ is closed;
\item $C$ is solid, meaning it has nonempty interior;
\item $C$ is pointed, meaning $x, -x\in C\Rightarrow x=0$.
\end{itemize}
\bigskip

A proper cone $C$ induces a partial ordering on $\mathbb{R}^n$:
\begin{displaymath}
a\preceq b\Leftrightarrow b-a\in C.
\end{displaymath}
This ordering has many nice properties, such as transitivity, reflexivity, and antisymmetry.
\par\bigskip

\begin{thebibliography}{4}
\bibitem{boyd} S. Boyd, L. Vandenberghe, \emph{Convex Optimization}, Cambridge University Press, 2004.
\end{thebibliography}
%%%%%
%%%%%
\end{document}
