\documentclass[12pt]{article}
\usepackage{pmmeta}
\pmcanonicalname{Polytope}
\pmcreated{2013-03-22 14:07:59}
\pmmodified{2013-03-22 14:07:59}
\pmowner{mps}{409}
\pmmodifier{mps}{409}
\pmtitle{polytope}
\pmrecord{26}{35544}
\pmprivacy{1}
\pmauthor{mps}{409}
\pmtype{Definition}
\pmcomment{trigger rebuild}
\pmclassification{msc}{52B40}
%\pmkeywords{polytope}
%\pmkeywords{face lattice}
%\pmkeywords{graded poset}
\pmrelated{Polyhedron}
\pmrelated{PoincareFormula}
\pmrelated{EulersPolyhedronTheorem}
\pmdefines{V-polytope}
\pmdefines{H-polytope}
\pmdefines{d-polytope}
\pmdefines{dimension}

\endmetadata

% this is the default PlanetMath preamble.  as your knowledge
% of TeX increases, you will probably want to edit this, but
% it should be fine as is for beginners.

% almost certainly you want these
\usepackage{amssymb}
\usepackage{amsmath}
\usepackage{amsfonts}

% used for TeXing text within eps files
%\usepackage{psfrag}
% need this for including graphics (\includegraphics)
%\usepackage{graphicx}
% for neatly defining theorems and propositions
%\usepackage{amsthm}
% making logically defined graphics
%%%\usepackage{xypic}

% there are many more packages, add them here as you need them

% define commands here
\begin{document}
A \emph{polytope} is the convex hull of finitely many points in
Euclidean space.  A polytope constructed in this way is the convex
hull of its vertices and is called a \emph{$\mathcal{V}$-polytope}.
An \emph{$\mathcal{H}$-polytope} is a bounded intersection of
upper halfspaces.  By the Weyl--Minkowski theorem, these descriptions are
equivalent, that is, every $\mathcal{V}$-polytope is an
$\mathcal{H}$-polytope, and vice versa.  This shows that our
intuition, based on the study of low-dimensional polytopes, that one
can describe a polytope either by its vertices or by its facets is
essentially correct.

The \emph{dimension} of $P$ is the smallest $d$ such that $P$ can be
embedded in $\mathbb{R}^d$.  A $d$-dimensional polytope is also called
a \emph{$d$-polytope}.

A face of a polytope is the intersection of the polytope with a
supporting hyperplane.  Intuitively, a supporting hyperplane is a
hyperplane that ``just touches'' the polytope, as though the polytope
were just about to pass through the hyperplane.  Note that this
intuitive picture does not cover the case of the empty face, where the
supporting hyperplane does not touch the polytope at all, or the fact
that a polytope is a face of itself.  The faces of a polytope, when 
partially ordered by set inclusion, form a geometric lattice, called 
the face lattice of the polytope.

The Euler polyhedron formula, which states that if a $3$-polytope has
$V$ vertices, $E$ edges, and $F$ faces, then
\[
V - E + F = 2,
\]
has a generalization to all $d$-polytopes.  Let $(f_{-1}=1, f_0,
\dots, f_{d-1}, f_d = 1)$ be the f-vector of a $d$-polytope $P$, so
$f_i$ is the number of $i$-dimensional faces of $P$.  Then these numbers
satsify the Euler--Poincar\'e--Schl\"afli formula:
\begin{equation}
\sum_{i=-1}^d (-1)^i f_i = 0.
\end{equation}
This is the first of many relations among entries of the f-vector
satisfied by all polytopes.  These relations are called the
\emph{Dehn--Sommerville relations}.  Any poset which satisfies these
relations is \PMlinkname{Eulerian}{EulerianPoset},
so the face lattice of any polytope is Eulerian.

\begin{thebibliography}{3}
\bibitem{cite:BB}
Bayer, M. and L. Billera, \emph{Generalized Dehn--Sommerville relations for
polytopes, spheres and Eulerian partially ordered sets}, Invent. Math. 79
(1985), no. 1, 143--157.
\bibitem{cite:BK}
Bayer, M. and A. Klapper, \emph{A new index for polytopes}, Discrete Comput.
Geom. 6
(1991), no. 1, 33--47.
\bibitem{cite:M}
Minkowski, H.  \emph{Allgemeine Lehrs\"atze \"uber die konvexe Polyeder}, Nachr.~Ges.~Wiss., G\"ottingen, 1897, 198--219.
\bibitem{cite:W}
Weyl, H.  \emph{Elementare Theorie der konvexen Polyeder}, Comment.~Math.~Helvetici, 1935, 7
\bibitem{cite:Z}
Ziegler, G., \emph{Lectures on polytopes}, Springer-Verlag, 1997.
\end{thebibliography}

\PMlinkescapeword{satisfy}
\PMlinkescapeword{inclusion}
\PMlinkescapeword{collection}
\PMlinkescapeword{dimension}
\PMlinkescapeword{dimensions}
\PMlinkescapeword{spheres}
\PMlinkescapeword{index}
\PMlinkescapeword{relation}
\PMlinkescapeword{relations}
\PMlinkescapeword{identities}
\PMlinkescapeword{meet}
\PMlinkescapeword{meets}
\PMlinkescapeword{join}
\PMlinkescapeword{lattice}
\PMlinkescapeword{extension}
\PMlinkescapeword{formula}
%%%%%
%%%%%
\end{document}
