\documentclass[12pt]{article}
\usepackage{pmmeta}
\pmcanonicalname{ProofOfCriterionForConvexityII}
\pmcreated{2013-03-22 18:25:25}
\pmmodified{2013-03-22 18:25:25}
\pmowner{yesitis}{13730}
\pmmodifier{yesitis}{13730}
\pmtitle{proof of criterion for convexity II}
\pmrecord{5}{41077}
\pmprivacy{1}
\pmauthor{yesitis}{13730}
\pmtype{Proof}
\pmcomment{trigger rebuild}
\pmclassification{msc}{52A41}
\pmclassification{msc}{26A51}
\pmclassification{msc}{26B25}

% this is the default PlanetMath preamble.  as your knowledge
% of TeX increases, you will probably want to edit this, but
% it should be fine as is for beginners.

% almost certainly you want these
\usepackage{amssymb}
\usepackage{amsmath}
\usepackage{amsfonts}

% used for TeXing text within eps files
%\usepackage{psfrag}
% need this for including graphics (\includegraphics)
%\usepackage{graphicx}
% for neatly defining theorems and propositions
%\usepackage{amsthm}
% making logically defined graphics
%%%\usepackage{xypic}

% there are many more packages, add them here as you need them

% define commands here

\begin{document}
If $f$ was not convex, then there was a point $\xi\in(a, b)$ such
that $f(\xi)>h(x)=\frac{f(v)-f(u)}{v-u}(x-u)+f(u)$ for some $u<v$ in
$(a,b)$. Since $f$ is continuous, there would be a neighborhood
$(\xi-\delta, \xi+\delta), \delta>0$, of $\xi$ such that $f(x)>h(x)$
for all $x$ in this neighborhood. (I.e., $f(x)$ was ``above" the
line segment joining $f(u)$ and $f(v)$.) Let $s=\xi-\delta,
t=\xi+\delta$.

Using the two points $A=(s, f(s)), B=(t, f(t))$, we construct
another line segment $\overline{AB}$ whose equation is given by
$g(x)=\frac{f(s)-f(t)}{2\delta}(x-s)+f(s)$; we have $f(x)>g(x)$ for
$x\in(s, t)$. In particular,
\begin{equation}
\displaystyle
f(\xi)=f\left(\frac{s+t}{2}\right)>g(\xi)=\frac{f(s)+f(t)}{2}.
\end{equation}
(One easily verifies $g(\xi)=(f(s)+f(t))/2$.) This contradicts
hypothesis.

Note that we have tacitly used the fact that $h(x)=\lambda
f(v)+(1-\lambda)f(u)$ for some $\lambda$ and $g(x)=\lambda
f(s)+(1-\lambda)f(t)$ for some $\lambda$.

%%%%%
%%%%%
\end{document}
