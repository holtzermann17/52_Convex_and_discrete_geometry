\documentclass[12pt]{article}
\usepackage{pmmeta}
\pmcanonicalname{IfAIsConvexAndFLinearThenFAAndF1AAreConvex}
\pmcreated{2013-03-22 14:36:18}
\pmmodified{2013-03-22 14:36:18}
\pmowner{matte}{1858}
\pmmodifier{matte}{1858}
\pmtitle{if $A$ is convex and $f$ linear then $f(A)$ and $f^{-1}(A)$ are convex}
\pmrecord{8}{36175}
\pmprivacy{1}
\pmauthor{matte}{1858}
\pmtype{Theorem}
\pmcomment{trigger rebuild}
\pmclassification{msc}{52A99}
\pmrelated{InverseImage}
\pmrelated{DirectImage}

\endmetadata

% this is the default PlanetMath preamble.  as your knowledge
% of TeX increases, you will probably want to edit this, but
% it should be fine as is for beginners.

% almost certainly you want these
\usepackage{amssymb}
\usepackage{amsmath}
\usepackage{amsfonts}
\usepackage{amsthm}

% used for TeXing text within eps files
%\usepackage{psfrag}
% need this for including graphics (\includegraphics)
%\usepackage{graphicx}
% for neatly defining theorems and propositions
%
% making logically defined graphics
%%%\usepackage{xypic}

% there are many more packages, add them here as you need them

% define commands here

\newcommand{\sR}[0]{\mathbb{R}}
\newcommand{\sC}[0]{\mathbb{C}}
\newcommand{\sN}[0]{\mathbb{N}}
\newcommand{\sZ}[0]{\mathbb{Z}}

 \usepackage{bbm}
 \newcommand{\Z}{\mathbbmss{Z}}
 \newcommand{\C}{\mathbbmss{C}}
 \newcommand{\R}{\mathbbmss{R}}
 \newcommand{\Q}{\mathbbmss{Q}}



\newcommand*{\norm}[1]{\lVert #1 \rVert}
\newcommand*{\abs}[1]{| #1 |}



\newtheorem{thm}{Theorem}
\newtheorem{defn}{Definition}
\newtheorem{prop}{Proposition}
\newtheorem{lemma}{Lemma}
\newtheorem{cor}{Corollary}
\begin{document}
\begin{prop}
Suppose $X$, $Y$ are vector spaces over $\sR$ (or $\sC$),
    and suppose $f\colon X\to Y$ is a linear map.
\begin{enumerate}
\item  If $A\subseteq X$ is convex, then $f(A)$ is convex.
\item  If $B\subseteq Y$ is convex, then $f^{-1}(B)$ is convex,
where $f^{-1}$ is the inverse image. 
\end{enumerate}
\end{prop}

\begin{proof} For the first claim, 
suppose $y,y'\in f(A)$, say,
$y=f(x)$ and $y'=f(x')$ for $x,x'\in A$, and suppose $\lambda\in (0,1)$.
Then 
\begin{eqnarray*}
  \lambda y + (1-\lambda)y' &=& \lambda f(x) + (1-\lambda ) f(x') \\
    &=& f( \lambda x + (1-\lambda) x'),
\end{eqnarray*}
so  $\lambda y + (1-\lambda)y' \in f(A)$ as $A$ is convex.

For the second claim, let us first recall that $x\in f^{-1}(B)$
if and only if $f(x)\in B$. Then, if $x,x'\in f^{-1}(B)$, 
and $\lambda\in (0,1)$, we have
\begin{eqnarray*}
f(\lambda x + (1-\lambda)x') &=& \lambda f(x) + (1-\lambda ) f(x').
\end{eqnarray*}
As $B$ is convex, the right hand side belongs to $B$, 
and $\lambda x + (1-\lambda)x' \in f^{-1}(B)$. 
\end{proof}
%%%%%
%%%%%
\end{document}
