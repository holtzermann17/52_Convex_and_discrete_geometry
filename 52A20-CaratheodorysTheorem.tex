\documentclass[12pt]{article}
\usepackage{pmmeta}
\pmcanonicalname{CaratheodorysTheorem}
\pmcreated{2013-03-22 13:57:43}
\pmmodified{2013-03-22 13:57:43}
\pmowner{bbukh}{348}
\pmmodifier{bbukh}{348}
\pmtitle{Carath\'eodory's theorem}
\pmrecord{6}{34730}
\pmprivacy{1}
\pmauthor{bbukh}{348}
\pmtype{Theorem}
\pmcomment{trigger rebuild}
\pmclassification{msc}{52A20}
\pmrelated{ConvexSet}

\usepackage{amssymb}
\usepackage{amsmath}
\usepackage{amsfonts}

% used for TeXing text within eps files
%\usepackage{psfrag}
% need this for including graphics (\includegraphics)
%\usepackage{graphicx}
% for neatly defining theorems and propositions
%\usepackage{amsthm}
% making logically defined graphics
%%%\usepackage{xypic}

\makeatletter
\@ifundefined{bibname}{}{\renewcommand{\bibname}{References}}
\makeatother
\begin{document}
Suppose a point $p$ lies in the convex hull of a set $P\subset \mathbb{R}^d$. Then there is a subset $P'\subset P$ consisting of no more than $d+1$ points such that $p$ lies in the convex hull of $P'$.

For example, if a point $p$ is contained in a convex hull of a set $P\subset \mathbb{R}^2$, then there are three points in $P$ that determine the triangle containing $p$, provided, of course, that $P$ contains at least three points.
%%%%%
%%%%%
\end{document}
